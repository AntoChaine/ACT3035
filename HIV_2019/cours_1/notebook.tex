
% Default to the notebook output style

    


% Inherit from the specified cell style.




    
\documentclass[11pt]{article}

    
    
    \usepackage[T1]{fontenc}
    % Nicer default font (+ math font) than Computer Modern for most use cases
    \usepackage{mathpazo}

    % Basic figure setup, for now with no caption control since it's done
    % automatically by Pandoc (which extracts ![](path) syntax from Markdown).
    \usepackage{graphicx}
    % We will generate all images so they have a width \maxwidth. This means
    % that they will get their normal width if they fit onto the page, but
    % are scaled down if they would overflow the margins.
    \makeatletter
    \def\maxwidth{\ifdim\Gin@nat@width>\linewidth\linewidth
    \else\Gin@nat@width\fi}
    \makeatother
    \let\Oldincludegraphics\includegraphics
    % Set max figure width to be 80% of text width, for now hardcoded.
    \renewcommand{\includegraphics}[1]{\Oldincludegraphics[width=.8\maxwidth]{#1}}
    % Ensure that by default, figures have no caption (until we provide a
    % proper Figure object with a Caption API and a way to capture that
    % in the conversion process - todo).
    \usepackage{caption}
    \DeclareCaptionLabelFormat{nolabel}{}
    \captionsetup{labelformat=nolabel}

    \usepackage{adjustbox} % Used to constrain images to a maximum size 
    \usepackage{xcolor} % Allow colors to be defined
    \usepackage{enumerate} % Needed for markdown enumerations to work
    \usepackage{geometry} % Used to adjust the document margins
    \usepackage{amsmath} % Equations
    \usepackage{amssymb} % Equations
    \usepackage{textcomp} % defines textquotesingle
    % Hack from http://tex.stackexchange.com/a/47451/13684:
    \AtBeginDocument{%
        \def\PYZsq{\textquotesingle}% Upright quotes in Pygmentized code
    }
    \usepackage{upquote} % Upright quotes for verbatim code
    \usepackage{eurosym} % defines \euro
    \usepackage[mathletters]{ucs} % Extended unicode (utf-8) support
    \usepackage[utf8x]{inputenc} % Allow utf-8 characters in the tex document
    \usepackage{fancyvrb} % verbatim replacement that allows latex
    \usepackage{grffile} % extends the file name processing of package graphics 
                         % to support a larger range 
    % The hyperref package gives us a pdf with properly built
    % internal navigation ('pdf bookmarks' for the table of contents,
    % internal cross-reference links, web links for URLs, etc.)
    \usepackage{hyperref}
    \usepackage{longtable} % longtable support required by pandoc >1.10
    \usepackage{booktabs}  % table support for pandoc > 1.12.2
    \usepackage[inline]{enumitem} % IRkernel/repr support (it uses the enumerate* environment)
    \usepackage[normalem]{ulem} % ulem is needed to support strikethroughs (\sout)
                                % normalem makes italics be italics, not underlines
    

    
    
    % Colors for the hyperref package
    \definecolor{urlcolor}{rgb}{0,.145,.698}
    \definecolor{linkcolor}{rgb}{.71,0.21,0.01}
    \definecolor{citecolor}{rgb}{.12,.54,.11}

    % ANSI colors
    \definecolor{ansi-black}{HTML}{3E424D}
    \definecolor{ansi-black-intense}{HTML}{282C36}
    \definecolor{ansi-red}{HTML}{E75C58}
    \definecolor{ansi-red-intense}{HTML}{B22B31}
    \definecolor{ansi-green}{HTML}{00A250}
    \definecolor{ansi-green-intense}{HTML}{007427}
    \definecolor{ansi-yellow}{HTML}{DDB62B}
    \definecolor{ansi-yellow-intense}{HTML}{B27D12}
    \definecolor{ansi-blue}{HTML}{208FFB}
    \definecolor{ansi-blue-intense}{HTML}{0065CA}
    \definecolor{ansi-magenta}{HTML}{D160C4}
    \definecolor{ansi-magenta-intense}{HTML}{A03196}
    \definecolor{ansi-cyan}{HTML}{60C6C8}
    \definecolor{ansi-cyan-intense}{HTML}{258F8F}
    \definecolor{ansi-white}{HTML}{C5C1B4}
    \definecolor{ansi-white-intense}{HTML}{A1A6B2}

    % commands and environments needed by pandoc snippets
    % extracted from the output of `pandoc -s`
    \providecommand{\tightlist}{%
      \setlength{\itemsep}{0pt}\setlength{\parskip}{0pt}}
    \DefineVerbatimEnvironment{Highlighting}{Verbatim}{commandchars=\\\{\}}
    % Add ',fontsize=\small' for more characters per line
    \newenvironment{Shaded}{}{}
    \newcommand{\KeywordTok}[1]{\textcolor[rgb]{0.00,0.44,0.13}{\textbf{{#1}}}}
    \newcommand{\DataTypeTok}[1]{\textcolor[rgb]{0.56,0.13,0.00}{{#1}}}
    \newcommand{\DecValTok}[1]{\textcolor[rgb]{0.25,0.63,0.44}{{#1}}}
    \newcommand{\BaseNTok}[1]{\textcolor[rgb]{0.25,0.63,0.44}{{#1}}}
    \newcommand{\FloatTok}[1]{\textcolor[rgb]{0.25,0.63,0.44}{{#1}}}
    \newcommand{\CharTok}[1]{\textcolor[rgb]{0.25,0.44,0.63}{{#1}}}
    \newcommand{\StringTok}[1]{\textcolor[rgb]{0.25,0.44,0.63}{{#1}}}
    \newcommand{\CommentTok}[1]{\textcolor[rgb]{0.38,0.63,0.69}{\textit{{#1}}}}
    \newcommand{\OtherTok}[1]{\textcolor[rgb]{0.00,0.44,0.13}{{#1}}}
    \newcommand{\AlertTok}[1]{\textcolor[rgb]{1.00,0.00,0.00}{\textbf{{#1}}}}
    \newcommand{\FunctionTok}[1]{\textcolor[rgb]{0.02,0.16,0.49}{{#1}}}
    \newcommand{\RegionMarkerTok}[1]{{#1}}
    \newcommand{\ErrorTok}[1]{\textcolor[rgb]{1.00,0.00,0.00}{\textbf{{#1}}}}
    \newcommand{\NormalTok}[1]{{#1}}
    
    % Additional commands for more recent versions of Pandoc
    \newcommand{\ConstantTok}[1]{\textcolor[rgb]{0.53,0.00,0.00}{{#1}}}
    \newcommand{\SpecialCharTok}[1]{\textcolor[rgb]{0.25,0.44,0.63}{{#1}}}
    \newcommand{\VerbatimStringTok}[1]{\textcolor[rgb]{0.25,0.44,0.63}{{#1}}}
    \newcommand{\SpecialStringTok}[1]{\textcolor[rgb]{0.73,0.40,0.53}{{#1}}}
    \newcommand{\ImportTok}[1]{{#1}}
    \newcommand{\DocumentationTok}[1]{\textcolor[rgb]{0.73,0.13,0.13}{\textit{{#1}}}}
    \newcommand{\AnnotationTok}[1]{\textcolor[rgb]{0.38,0.63,0.69}{\textbf{\textit{{#1}}}}}
    \newcommand{\CommentVarTok}[1]{\textcolor[rgb]{0.38,0.63,0.69}{\textbf{\textit{{#1}}}}}
    \newcommand{\VariableTok}[1]{\textcolor[rgb]{0.10,0.09,0.49}{{#1}}}
    \newcommand{\ControlFlowTok}[1]{\textcolor[rgb]{0.00,0.44,0.13}{\textbf{{#1}}}}
    \newcommand{\OperatorTok}[1]{\textcolor[rgb]{0.40,0.40,0.40}{{#1}}}
    \newcommand{\BuiltInTok}[1]{{#1}}
    \newcommand{\ExtensionTok}[1]{{#1}}
    \newcommand{\PreprocessorTok}[1]{\textcolor[rgb]{0.74,0.48,0.00}{{#1}}}
    \newcommand{\AttributeTok}[1]{\textcolor[rgb]{0.49,0.56,0.16}{{#1}}}
    \newcommand{\InformationTok}[1]{\textcolor[rgb]{0.38,0.63,0.69}{\textbf{\textit{{#1}}}}}
    \newcommand{\WarningTok}[1]{\textcolor[rgb]{0.38,0.63,0.69}{\textbf{\textit{{#1}}}}}
    
    
    % Define a nice break command that doesn't care if a line doesn't already
    % exist.
    \def\br{\hspace*{\fill} \\* }
    % Math Jax compatability definitions
    \def\gt{>}
    \def\lt{<}
    % Document parameters
    \title{1\_1\_cours}
    
    
    

    % Pygments definitions
    
\makeatletter
\def\PY@reset{\let\PY@it=\relax \let\PY@bf=\relax%
    \let\PY@ul=\relax \let\PY@tc=\relax%
    \let\PY@bc=\relax \let\PY@ff=\relax}
\def\PY@tok#1{\csname PY@tok@#1\endcsname}
\def\PY@toks#1+{\ifx\relax#1\empty\else%
    \PY@tok{#1}\expandafter\PY@toks\fi}
\def\PY@do#1{\PY@bc{\PY@tc{\PY@ul{%
    \PY@it{\PY@bf{\PY@ff{#1}}}}}}}
\def\PY#1#2{\PY@reset\PY@toks#1+\relax+\PY@do{#2}}

\expandafter\def\csname PY@tok@w\endcsname{\def\PY@tc##1{\textcolor[rgb]{0.73,0.73,0.73}{##1}}}
\expandafter\def\csname PY@tok@c\endcsname{\let\PY@it=\textit\def\PY@tc##1{\textcolor[rgb]{0.25,0.50,0.50}{##1}}}
\expandafter\def\csname PY@tok@cp\endcsname{\def\PY@tc##1{\textcolor[rgb]{0.74,0.48,0.00}{##1}}}
\expandafter\def\csname PY@tok@k\endcsname{\let\PY@bf=\textbf\def\PY@tc##1{\textcolor[rgb]{0.00,0.50,0.00}{##1}}}
\expandafter\def\csname PY@tok@kp\endcsname{\def\PY@tc##1{\textcolor[rgb]{0.00,0.50,0.00}{##1}}}
\expandafter\def\csname PY@tok@kt\endcsname{\def\PY@tc##1{\textcolor[rgb]{0.69,0.00,0.25}{##1}}}
\expandafter\def\csname PY@tok@o\endcsname{\def\PY@tc##1{\textcolor[rgb]{0.40,0.40,0.40}{##1}}}
\expandafter\def\csname PY@tok@ow\endcsname{\let\PY@bf=\textbf\def\PY@tc##1{\textcolor[rgb]{0.67,0.13,1.00}{##1}}}
\expandafter\def\csname PY@tok@nb\endcsname{\def\PY@tc##1{\textcolor[rgb]{0.00,0.50,0.00}{##1}}}
\expandafter\def\csname PY@tok@nf\endcsname{\def\PY@tc##1{\textcolor[rgb]{0.00,0.00,1.00}{##1}}}
\expandafter\def\csname PY@tok@nc\endcsname{\let\PY@bf=\textbf\def\PY@tc##1{\textcolor[rgb]{0.00,0.00,1.00}{##1}}}
\expandafter\def\csname PY@tok@nn\endcsname{\let\PY@bf=\textbf\def\PY@tc##1{\textcolor[rgb]{0.00,0.00,1.00}{##1}}}
\expandafter\def\csname PY@tok@ne\endcsname{\let\PY@bf=\textbf\def\PY@tc##1{\textcolor[rgb]{0.82,0.25,0.23}{##1}}}
\expandafter\def\csname PY@tok@nv\endcsname{\def\PY@tc##1{\textcolor[rgb]{0.10,0.09,0.49}{##1}}}
\expandafter\def\csname PY@tok@no\endcsname{\def\PY@tc##1{\textcolor[rgb]{0.53,0.00,0.00}{##1}}}
\expandafter\def\csname PY@tok@nl\endcsname{\def\PY@tc##1{\textcolor[rgb]{0.63,0.63,0.00}{##1}}}
\expandafter\def\csname PY@tok@ni\endcsname{\let\PY@bf=\textbf\def\PY@tc##1{\textcolor[rgb]{0.60,0.60,0.60}{##1}}}
\expandafter\def\csname PY@tok@na\endcsname{\def\PY@tc##1{\textcolor[rgb]{0.49,0.56,0.16}{##1}}}
\expandafter\def\csname PY@tok@nt\endcsname{\let\PY@bf=\textbf\def\PY@tc##1{\textcolor[rgb]{0.00,0.50,0.00}{##1}}}
\expandafter\def\csname PY@tok@nd\endcsname{\def\PY@tc##1{\textcolor[rgb]{0.67,0.13,1.00}{##1}}}
\expandafter\def\csname PY@tok@s\endcsname{\def\PY@tc##1{\textcolor[rgb]{0.73,0.13,0.13}{##1}}}
\expandafter\def\csname PY@tok@sd\endcsname{\let\PY@it=\textit\def\PY@tc##1{\textcolor[rgb]{0.73,0.13,0.13}{##1}}}
\expandafter\def\csname PY@tok@si\endcsname{\let\PY@bf=\textbf\def\PY@tc##1{\textcolor[rgb]{0.73,0.40,0.53}{##1}}}
\expandafter\def\csname PY@tok@se\endcsname{\let\PY@bf=\textbf\def\PY@tc##1{\textcolor[rgb]{0.73,0.40,0.13}{##1}}}
\expandafter\def\csname PY@tok@sr\endcsname{\def\PY@tc##1{\textcolor[rgb]{0.73,0.40,0.53}{##1}}}
\expandafter\def\csname PY@tok@ss\endcsname{\def\PY@tc##1{\textcolor[rgb]{0.10,0.09,0.49}{##1}}}
\expandafter\def\csname PY@tok@sx\endcsname{\def\PY@tc##1{\textcolor[rgb]{0.00,0.50,0.00}{##1}}}
\expandafter\def\csname PY@tok@m\endcsname{\def\PY@tc##1{\textcolor[rgb]{0.40,0.40,0.40}{##1}}}
\expandafter\def\csname PY@tok@gh\endcsname{\let\PY@bf=\textbf\def\PY@tc##1{\textcolor[rgb]{0.00,0.00,0.50}{##1}}}
\expandafter\def\csname PY@tok@gu\endcsname{\let\PY@bf=\textbf\def\PY@tc##1{\textcolor[rgb]{0.50,0.00,0.50}{##1}}}
\expandafter\def\csname PY@tok@gd\endcsname{\def\PY@tc##1{\textcolor[rgb]{0.63,0.00,0.00}{##1}}}
\expandafter\def\csname PY@tok@gi\endcsname{\def\PY@tc##1{\textcolor[rgb]{0.00,0.63,0.00}{##1}}}
\expandafter\def\csname PY@tok@gr\endcsname{\def\PY@tc##1{\textcolor[rgb]{1.00,0.00,0.00}{##1}}}
\expandafter\def\csname PY@tok@ge\endcsname{\let\PY@it=\textit}
\expandafter\def\csname PY@tok@gs\endcsname{\let\PY@bf=\textbf}
\expandafter\def\csname PY@tok@gp\endcsname{\let\PY@bf=\textbf\def\PY@tc##1{\textcolor[rgb]{0.00,0.00,0.50}{##1}}}
\expandafter\def\csname PY@tok@go\endcsname{\def\PY@tc##1{\textcolor[rgb]{0.53,0.53,0.53}{##1}}}
\expandafter\def\csname PY@tok@gt\endcsname{\def\PY@tc##1{\textcolor[rgb]{0.00,0.27,0.87}{##1}}}
\expandafter\def\csname PY@tok@err\endcsname{\def\PY@bc##1{\setlength{\fboxsep}{0pt}\fcolorbox[rgb]{1.00,0.00,0.00}{1,1,1}{\strut ##1}}}
\expandafter\def\csname PY@tok@kc\endcsname{\let\PY@bf=\textbf\def\PY@tc##1{\textcolor[rgb]{0.00,0.50,0.00}{##1}}}
\expandafter\def\csname PY@tok@kd\endcsname{\let\PY@bf=\textbf\def\PY@tc##1{\textcolor[rgb]{0.00,0.50,0.00}{##1}}}
\expandafter\def\csname PY@tok@kn\endcsname{\let\PY@bf=\textbf\def\PY@tc##1{\textcolor[rgb]{0.00,0.50,0.00}{##1}}}
\expandafter\def\csname PY@tok@kr\endcsname{\let\PY@bf=\textbf\def\PY@tc##1{\textcolor[rgb]{0.00,0.50,0.00}{##1}}}
\expandafter\def\csname PY@tok@bp\endcsname{\def\PY@tc##1{\textcolor[rgb]{0.00,0.50,0.00}{##1}}}
\expandafter\def\csname PY@tok@fm\endcsname{\def\PY@tc##1{\textcolor[rgb]{0.00,0.00,1.00}{##1}}}
\expandafter\def\csname PY@tok@vc\endcsname{\def\PY@tc##1{\textcolor[rgb]{0.10,0.09,0.49}{##1}}}
\expandafter\def\csname PY@tok@vg\endcsname{\def\PY@tc##1{\textcolor[rgb]{0.10,0.09,0.49}{##1}}}
\expandafter\def\csname PY@tok@vi\endcsname{\def\PY@tc##1{\textcolor[rgb]{0.10,0.09,0.49}{##1}}}
\expandafter\def\csname PY@tok@vm\endcsname{\def\PY@tc##1{\textcolor[rgb]{0.10,0.09,0.49}{##1}}}
\expandafter\def\csname PY@tok@sa\endcsname{\def\PY@tc##1{\textcolor[rgb]{0.73,0.13,0.13}{##1}}}
\expandafter\def\csname PY@tok@sb\endcsname{\def\PY@tc##1{\textcolor[rgb]{0.73,0.13,0.13}{##1}}}
\expandafter\def\csname PY@tok@sc\endcsname{\def\PY@tc##1{\textcolor[rgb]{0.73,0.13,0.13}{##1}}}
\expandafter\def\csname PY@tok@dl\endcsname{\def\PY@tc##1{\textcolor[rgb]{0.73,0.13,0.13}{##1}}}
\expandafter\def\csname PY@tok@s2\endcsname{\def\PY@tc##1{\textcolor[rgb]{0.73,0.13,0.13}{##1}}}
\expandafter\def\csname PY@tok@sh\endcsname{\def\PY@tc##1{\textcolor[rgb]{0.73,0.13,0.13}{##1}}}
\expandafter\def\csname PY@tok@s1\endcsname{\def\PY@tc##1{\textcolor[rgb]{0.73,0.13,0.13}{##1}}}
\expandafter\def\csname PY@tok@mb\endcsname{\def\PY@tc##1{\textcolor[rgb]{0.40,0.40,0.40}{##1}}}
\expandafter\def\csname PY@tok@mf\endcsname{\def\PY@tc##1{\textcolor[rgb]{0.40,0.40,0.40}{##1}}}
\expandafter\def\csname PY@tok@mh\endcsname{\def\PY@tc##1{\textcolor[rgb]{0.40,0.40,0.40}{##1}}}
\expandafter\def\csname PY@tok@mi\endcsname{\def\PY@tc##1{\textcolor[rgb]{0.40,0.40,0.40}{##1}}}
\expandafter\def\csname PY@tok@il\endcsname{\def\PY@tc##1{\textcolor[rgb]{0.40,0.40,0.40}{##1}}}
\expandafter\def\csname PY@tok@mo\endcsname{\def\PY@tc##1{\textcolor[rgb]{0.40,0.40,0.40}{##1}}}
\expandafter\def\csname PY@tok@ch\endcsname{\let\PY@it=\textit\def\PY@tc##1{\textcolor[rgb]{0.25,0.50,0.50}{##1}}}
\expandafter\def\csname PY@tok@cm\endcsname{\let\PY@it=\textit\def\PY@tc##1{\textcolor[rgb]{0.25,0.50,0.50}{##1}}}
\expandafter\def\csname PY@tok@cpf\endcsname{\let\PY@it=\textit\def\PY@tc##1{\textcolor[rgb]{0.25,0.50,0.50}{##1}}}
\expandafter\def\csname PY@tok@c1\endcsname{\let\PY@it=\textit\def\PY@tc##1{\textcolor[rgb]{0.25,0.50,0.50}{##1}}}
\expandafter\def\csname PY@tok@cs\endcsname{\let\PY@it=\textit\def\PY@tc##1{\textcolor[rgb]{0.25,0.50,0.50}{##1}}}

\def\PYZbs{\char`\\}
\def\PYZus{\char`\_}
\def\PYZob{\char`\{}
\def\PYZcb{\char`\}}
\def\PYZca{\char`\^}
\def\PYZam{\char`\&}
\def\PYZlt{\char`\<}
\def\PYZgt{\char`\>}
\def\PYZsh{\char`\#}
\def\PYZpc{\char`\%}
\def\PYZdl{\char`\$}
\def\PYZhy{\char`\-}
\def\PYZsq{\char`\'}
\def\PYZdq{\char`\"}
\def\PYZti{\char`\~}
% for compatibility with earlier versions
\def\PYZat{@}
\def\PYZlb{[}
\def\PYZrb{]}
\makeatother


    % Exact colors from NB
    \definecolor{incolor}{rgb}{0.0, 0.0, 0.5}
    \definecolor{outcolor}{rgb}{0.545, 0.0, 0.0}



    
    % Prevent overflowing lines due to hard-to-break entities
    \sloppy 
    % Setup hyperref package
    \hypersetup{
      breaklinks=true,  % so long urls are correctly broken across lines
      colorlinks=true,
      urlcolor=urlcolor,
      linkcolor=linkcolor,
      citecolor=citecolor,
      }
    % Slightly bigger margins than the latex defaults
    
    \geometry{verbose,tmargin=1in,bmargin=1in,lmargin=1in,rmargin=1in}
    
    

    \begin{document}
    
    
    \maketitle
    
    

    
    Table of Contents{}

{{1~~}Ouvrir Rstudio}

{{1.1~~}Sous Linux}

{{1.2~~}Sous macOS}

{{1.3~~}Sous windows,}

{{2~~}Présentation de RStudio}

{{2.1~~}Les sous-section}

{{2.2~~}Créer un script}

{{3~~}Packages}

{{3.1~~}autocompletion}

{{3.2~~}Quels packages sont chargé dans l'environnement?}

{{4~~}Assignation des variables}

{{4.1~~}avec le signe =}

{{4.2~~}La flèche vers la gauche \textless{}-}

{{4.3~~}La flèche vers la droite -\textgreater{}}

{{4.4~~}La fonction assign()}

{{5~~}Les nombres, les caractères et les booléens}

{{5.1~~}Changement du type de variable}

{{5.2~~}Tester le type de variable}

{{5.2.1~~}Test d'égalité}

{{5.2.2~~}Test d'inégalité}

{{6~~}Les opérations sur le workspace}

{{6.1~~}Répertoire courant}

{{6.2~~}Changer le répertoire courant}

{{6.3~~}Lister les objets dans la mémoire}

{{6.4~~}Supprimer un objets de la mémoire}

{{6.5~~}Vider complètement le workspace}

{{7~~}Help}

{{7.1~~}Aide sur les fonctions}

{{7.2~~}La fonction example}

{{7.3~~}Aide sur les données}

{{8~~}Plus de resources}

    Dans ce cours, nous allons utiliser la console RStudio que nous avons
installé. Nous allons présenter la console, et découvrir les principales
sections que nous utiliserons tout au long des prochains cours sur la
programmation en R.

    \hypertarget{ouvrir-rstudio}{%
\section{Ouvrir Rstudio}\label{ouvrir-rstudio}}

    \hypertarget{sous-linux}{%
\subsection{Sous Linux}\label{sous-linux}}

vous ouvrez le terminal en appuyant sur les touches \texttt{Ctrl+Alt+T}
et vous tapez \texttt{RStudio}

    \hypertarget{sous-macos}{%
\subsection{Sous macOS}\label{sous-macos}}

Vous appuyez sur la touche \textbf{\texttt{cmd+space}} vous tapez
ensuite sur \texttt{RSdtudio} en \texttt{Enter} ensuite

    \hypertarget{sous-windows}{%
\subsection{Sous windows,}\label{sous-windows}}

Cherchez l'application RStudio et vous cliquez là-dessus pour la lancer

    \hypertarget{pruxe9sentation-de-rstudio}{%
\section{Présentation de RStudio}\label{pruxe9sentation-de-rstudio}}

    Lorsque nous ouvrons RStudion, nous avons alors une fenêtre avec trois
sections dans lesquels nous allons travailler.

    

    \hypertarget{les-sous-section}{%
\subsection{Les sous-section}\label{les-sous-section}}

    \begin{enumerate}
\def\labelenumi{\arabic{enumi}.}
\tightlist
\item
  La console est là où le code R est exécuté, cette console est la même
  que ce que nous voyons si nous ouvrons R
\item
  Dans la deuxième section, nous retrouvons;

  \begin{enumerate}
  \def\labelenumii{\arabic{enumii}.}
  \tightlist
  \item
    \emph{Environment}, dans ce dernier nous retrouvons la liste des
    données ou objets à notre disposition. Par exemple, lorsque nous
    avons assigné la valeur 2 à la variable \(y\), nous remarquons alors
    que RStudio garde en mémoire la valeur de cette variable.
  \item
    History: cette sous-section contient l'historique des commandes que
    nous avons exécutées. Si nous cliquons sur une ligne quelconque,
    nous verrons alors cette ligne s'écrire dans la console. Lorsque
    nous appuyons sur \texttt{Enter}, la ligne s'exécute.
  \end{enumerate}
\item
  La troisième section contient;

  \begin{enumerate}
  \def\labelenumii{\arabic{enumii}.}
  \tightlist
  \item
    Files: ici, nous pouvons naviguer directement dans le dossier dans
    lequel nous voulons; exécuter du code, créer des données, importer
    des données\ldots{}etc. Nous verrons un peu plus loin plus en détail
    cette sous-section.
  \item
    Plots: dans cette sous-section, nous retrouvons nos graphiques que
    nous avons exécutés, nous pouvons les exporter directement à partir
    de là
  \item
    Packages: Ici, nous retrouvons les packages qui nous sont
    disponibles à télécharger ou qui le sont déjà (coché ou pas).
  \item
    Help: Cette sous-section est très importante, car elle nous permet
    trouver la documentation du langage R. Nous avons qu'à écrire ce que
    nous cherchons.
  \end{enumerate}
\end{enumerate}

    \hypertarget{cruxe9er-un-script}{%
\subsection{Créer un script}\label{cruxe9er-un-script}}

    Afin de sauvegarder notre code R, nous pouvons ouvrir un nouveau script
en cliquant sur le petit signe plus vert en haut à gauche. Ou simplement
\textbf{\texttt{Ctrl+Shift+N}}

    Un script est simplement un éditeur de code R, dans lequel nous pouvons
exécuter notre code ligne par ligne en appuyant sur
\textbf{\texttt{Ctrl+Enter}} ou en cliquant sur le bouton \texttt{Run}

    

    Une fois votre script est créé, vous pouvez le sauvegarder à l'endroit
que vous voulez.

    \hypertarget{packages}{%
\section{Packages}\label{packages}}

    Les packages R sont une partie importante, ce sont une collection de
fonctions et de données créées par des individus (chercheurs, étudiants,
des geeks\ldots{}etc.) Au sein de la communauté open source. Lorsque
nous installons R, un ensemble de packages est déjà inclus.

Afin de savoir quels packages sont installés, il suffit de taper
\texttt{library()}

    \begin{Verbatim}[commandchars=\\\{\}]
{\color{incolor}In [{\color{incolor}1}]:} \PY{k+kn}{library}\PY{p}{(}\PY{p}{)}
\end{Verbatim}


    
    
    Lorsqu'on exécute une ligne de code, comme ce qu'on vient de faire avec
\texttt{library()}, on demande à \texttt{R} de trouver la fonction et de
l'exécuter. Habituellement, les fonctions requirent un argument, cette
fonction est une exception et ne requiert aucun argument. Ainsi, il
suffit de taper \texttt{library} sans les parenthèses.

    \begin{Verbatim}[commandchars=\\\{\}]
{\color{incolor}In [{\color{incolor}2}]:} \PY{c+c1}{\PYZsh{} library}
\end{Verbatim}


    \hypertarget{autocompletion}{%
\subsection{\texorpdfstring{\emph{autocompletion}}{autocompletion}}\label{autocompletion}}

    Dans RStudio, et dans la majorité des IDE récents, il existe une option
très utile appelée
\href{https://support.rstudio.com/hc/en-us/articles/205273297-Code-Completion}{\emph{autocompletion}}.
Elle devient très utile lorsque nous nous rappelons plus comment s'écrit
exactement une option ou quels en sont les arguments.

Il suffit de taper sur la touche \texttt{tab}

    Exemple: si nous voulons écrire la fonction \texttt{library()}, il
suffit d'écrire \texttt{lib} et taper sur la touche \texttt{tab}.
RStudio nous donne plusieurs choix de fonction que sont nom commence par
les trois lettres \texttt{lib}

    \begin{Verbatim}[commandchars=\\\{\}]
{\color{incolor}In [{\color{incolor}3}]:} lib
\end{Verbatim}


    \begin{Verbatim}[commandchars=\\\{\}]

        Error in eval(expr, envir, enclos): object 'lib' not found
    Traceback:


    \end{Verbatim}

    Une fois que le mot complet est saisi, en ouvrant des parenthèses, on
peut encore taper sur la touche \texttt{tab} afin d'avoir la liste des
arguments obligatoires ou optionnels à saisir.

    \begin{Verbatim}[commandchars=\\\{\}]
{\color{incolor}In [{\color{incolor}4}]:} \PY{k+kn}{library}\PY{p}{(}\PY{p}{)}
\end{Verbatim}


    
    
    \hypertarget{quels-packages-sont-charguxe9-dans-lenvironnement}{%
\subsection{Quels packages sont chargé dans
l'environnement?}\label{quels-packages-sont-charguxe9-dans-lenvironnement}}

    Afin de voir quels \_packagers sont chargés dans l'environnement, il
suffit de taper la commande \texttt{search()}

    \begin{Verbatim}[commandchars=\\\{\}]
{\color{incolor}In [{\color{incolor}5}]:} \PY{k+kp}{search}\PY{p}{(}\PY{p}{)}
\end{Verbatim}


    \begin{enumerate*}
\item '.GlobalEnv'
\item 'jupyter:irkernel'
\item 'package:stats'
\item 'package:graphics'
\item 'package:grDevices'
\item 'package:utils'
\item 'package:datasets'
\item 'package:methods'
\item 'Autoloads'
\item 'package:base'
\end{enumerate*}


    
    Un \emph{package} que nous utiliserons beaucoup au début est le
\_package \texttt{MASS}. Afin que nous soyons sûrs de l'avoir, nous
l'installons à nouveau avec la commande suivante:

    \begin{Verbatim}[commandchars=\\\{\}]
{\color{incolor}In [{\color{incolor} }]:} \PY{c+c1}{\PYZsh{} install.packages(\PYZdq{}MASS\PYZdq{})}
\end{Verbatim}


    Nous pouvons voir dans l'onglet \emph{packages} dans RSudio que nous
l'avons et qu'il prêt à charger. Nous procédons au chargement de ce
\emph{package} on le cochant ou simplement (il faut vraiment s'habituer
à travailler avec les commandes dans la console) avec la commande
suivante:

    On peut aussi installer un \emph{package} avec l'IDE (Integrated
development environment ) de RStudio

    \begin{Verbatim}[commandchars=\\\{\}]
{\color{incolor}In [{\color{incolor} }]:} \PY{c+c1}{\PYZsh{} require(MASS)}
\end{Verbatim}


    \textbf{Attention}: R est sensible aux caractères \emph{case sensitive}.
Donc si j'écris:

    \begin{Verbatim}[commandchars=\\\{\}]
{\color{incolor}In [{\color{incolor} }]:} \PY{c+c1}{\PYZsh{} require(mass)}
\end{Verbatim}


    J'ai alors un message d'erreur
\textbf{\texttt{“there\ is\ no\ package\ called\ ‘mass’”}}

    \hypertarget{assignation-des-variables}{%
\section{Assignation des variables}\label{assignation-des-variables}}

    Comment peut-on assigner des valeurs à des variables? R reconnait les
valeurs numériques telles qu'elles sont.

    \begin{Verbatim}[commandchars=\\\{\}]
{\color{incolor}In [{\color{incolor}6}]:} \PY{l+m}{3}
\end{Verbatim}


    3

    
    \begin{Verbatim}[commandchars=\\\{\}]
{\color{incolor}In [{\color{incolor}7}]:} \PY{l+m}{2}
\end{Verbatim}


    2

    
    Toutesfois, il existe d'autres variables numériques écrites en
caractère. Par exemple, \(\pi\). lorsqu'on saisit \texttt{pi} dans R, il
nous redonne la valeur numérique (arrondie) de \(\pi\)

    \begin{Verbatim}[commandchars=\\\{\}]
{\color{incolor}In [{\color{incolor}8}]:} \PY{k+kc}{pi}
\end{Verbatim}


    3.14159265358979

    
    Ce sont des variables déjà existantes dans \texttt{R}. Si l'on voulait
chercher des valeurs non existantes dans R, ce dernier nous retourne un
message d'erreur;

    \begin{Verbatim}[commandchars=\\\{\}]
{\color{incolor}In [{\color{incolor}9}]:} x
\end{Verbatim}


    \begin{Verbatim}[commandchars=\\\{\}]

        Error in eval(expr, envir, enclos): object 'x' not found
    Traceback:


    \end{Verbatim}

    On peut assigner une valeur à une variable de différentes manières;

    \hypertarget{avec-le-signe}{%
\subsection{avec le signe =}\label{avec-le-signe}}

    \begin{Verbatim}[commandchars=\\\{\}]
{\color{incolor}In [{\color{incolor}10}]:} x\PY{o}{=}\PY{l+m}{2}
         x
\end{Verbatim}


    2

    
    \begin{Verbatim}[commandchars=\\\{\}]
{\color{incolor}In [{\color{incolor}11}]:} y\PY{o}{=}\PY{l+m}{3}
         y
\end{Verbatim}


    3

    
    \hypertarget{la-fluxe8che-vers-la-gauche--}{%
\subsection{La flèche vers la gauche
\textless{}-}\label{la-fluxe8che-vers-la-gauche--}}

Par convention, nous utilison cette méthode

    \begin{Verbatim}[commandchars=\\\{\}]
{\color{incolor}In [{\color{incolor}12}]:} x\PY{o}{\PYZlt{}\PYZhy{}}\PY{l+m}{2}
         x
\end{Verbatim}


    2

    
    \begin{Verbatim}[commandchars=\\\{\}]
{\color{incolor}In [{\color{incolor}13}]:} y\PY{o}{\PYZlt{}\PYZhy{}}\PY{l+m}{3}
\end{Verbatim}


    Nous avons donné la valeur 2 à \(x\) et la valeur 3 à \(y\). Ces valeurs
sont gardées en mémoire, on peut d'ailleurs faire des opérations
mathématiques sur ces valeurs. Par exemple on veut \[x+y\]

    \begin{Verbatim}[commandchars=\\\{\}]
{\color{incolor}In [{\color{incolor}14}]:} x\PY{o}{+}y
\end{Verbatim}


    5

    
    Si l'on voulait appliquer, un calcule sur une variable que nous n'y
avons pas assigner une valeur auparavant, cela ne fonctionnerait pas,
puisque R ne l'a pas gardé en mémoire.

    \begin{Verbatim}[commandchars=\\\{\}]
{\color{incolor}In [{\color{incolor}15}]:} z\PY{l+m}{+1}
\end{Verbatim}


    \begin{Verbatim}[commandchars=\\\{\}]

        Error in eval(expr, envir, enclos): object 'z' not found
    Traceback:


    \end{Verbatim}

    Rappelons-nous que R est \emph{case sensitive}, par exemple:

    \begin{Verbatim}[commandchars=\\\{\}]
{\color{incolor}In [{\color{incolor}16}]:} X
\end{Verbatim}


    \begin{Verbatim}[commandchars=\\\{\}]

        Error in eval(expr, envir, enclos): object 'X' not found
    Traceback:


    \end{Verbatim}

    Nous avons essayé d'appeler \(X\), mais il nous retourne un message
d'erreur. Cela est dû à cause la majuscule.

    nous avions donné à x la valeur \texttt{x\textless{}-2} et
\texttt{y\textless{}-3}

    \begin{Verbatim}[commandchars=\\\{\}]
{\color{incolor}In [{\color{incolor}17}]:} \PY{k+kp}{print}\PY{p}{(}x\PY{p}{)}
         \PY{k+kp}{print}\PY{p}{(}y\PY{p}{)}
\end{Verbatim}


    \begin{Verbatim}[commandchars=\\\{\}]
[1] 2
[1] 3

    \end{Verbatim}

    Nous pouvons écraser la valeur de x en lui assignant la valeur de y;

    \begin{Verbatim}[commandchars=\\\{\}]
{\color{incolor}In [{\color{incolor}18}]:} x\PY{o}{\PYZlt{}\PYZhy{}}y
\end{Verbatim}


    \begin{Verbatim}[commandchars=\\\{\}]
{\color{incolor}In [{\color{incolor}19}]:} \PY{k+kp}{print}\PY{p}{(}x\PY{p}{)}
         \PY{k+kp}{print}\PY{p}{(}y\PY{p}{)}
\end{Verbatim}


    \begin{Verbatim}[commandchars=\\\{\}]
[1] 3
[1] 3

    \end{Verbatim}

    Soit maintenant \(z=9\), on peut aussi faire ceci:

    \begin{Verbatim}[commandchars=\\\{\}]
{\color{incolor}In [{\color{incolor}20}]:} x\PY{o}{\PYZlt{}\PYZhy{}}y\PY{o}{\PYZlt{}\PYZhy{}}z\PY{o}{\PYZlt{}\PYZhy{}}\PY{l+m}{9}
\end{Verbatim}


    \begin{Verbatim}[commandchars=\\\{\}]
{\color{incolor}In [{\color{incolor}21}]:} \PY{k+kp}{print}\PY{p}{(}x\PY{p}{)}
         \PY{k+kp}{print}\PY{p}{(}y\PY{p}{)}
         \PY{k+kp}{print}\PY{p}{(}z\PY{p}{)}
\end{Verbatim}


    \begin{Verbatim}[commandchars=\\\{\}]
[1] 9
[1] 9
[1] 9

    \end{Verbatim}

    Remarquez que R commence toujours par la fin, la valeur 9 a été assignée
à toutes les variables.

    \hypertarget{la-fluxe8che-vers-la-droite--}{%
\subsection{La flèche vers la droite
-\textgreater{}}\label{la-fluxe8che-vers-la-droite--}}

On peut aussi utiliser l'autre sens de la flèche (vers la droite) pour
assigner des valeurs à des variables

    \begin{Verbatim}[commandchars=\\\{\}]
{\color{incolor}In [{\color{incolor}22}]:} \PY{l+m}{15} \PY{o}{\PYZhy{}\PYZgt{}}p
\end{Verbatim}


    Toutefois, nous restons dans la convention et utilisons la flèche vers
la gauche

    \hypertarget{la-fonction-assign}{%
\subsection{\texorpdfstring{La fonction
\texttt{assign()}}{La fonction assign()}}\label{la-fonction-assign}}

    Nous pouvons aussi utliliser la fonction \texttt{assign()}

\begin{Shaded}
\begin{Highlighting}[]
\KeywordTok{assign}\NormalTok{(x, value, }\DataTypeTok{pos =} \DecValTok{-1}\NormalTok{, }\DataTypeTok{envir =} \KeywordTok{as.environment}\NormalTok{(pos),}
       \DataTypeTok{inherits =} \OtherTok{FALSE}\NormalTok{, }\DataTypeTok{immediate =} \OtherTok{TRUE}\NormalTok{)}
\end{Highlighting}
\end{Shaded}

    \begin{Verbatim}[commandchars=\\\{\}]
{\color{incolor}In [{\color{incolor}23}]:} \PY{k+kp}{assign}\PY{p}{(}\PY{l+s}{\PYZdq{}}\PY{l+s}{q\PYZdq{}}\PY{p}{,}\PY{l+m}{30}\PY{p}{)}
\end{Verbatim}


    \begin{Verbatim}[commandchars=\\\{\}]
{\color{incolor}In [{\color{incolor}24}]:} \PY{k+kp}{q}
\end{Verbatim}


    30

    
    Cette fonction est souvent utilisée à l'intérieur d'une boucle où l'on
voudrait assigner une valeur quelconque à une variable qui peut changer
lors des itérations

    \hypertarget{les-nombres-les-caractuxe8res-et-les-booluxe9ens}{%
\section{Les nombres, les caractères et les
booléens}\label{les-nombres-les-caractuxe8res-et-les-booluxe9ens}}

    \begin{Verbatim}[commandchars=\\\{\}]
{\color{incolor}In [{\color{incolor}25}]:} num\PY{o}{\PYZlt{}\PYZhy{}}\PY{l+m}{25}
         string\PY{o}{\PYZlt{}\PYZhy{}}\PY{l+s}{\PYZdq{}}\PY{l+s}{bonjour\PYZdq{}}
         booleen\PY{o}{\PYZlt{}\PYZhy{}}\PY{k+kc}{TRUE}
\end{Verbatim}


    On peut appeler la variable \texttt{string} et elle nous retourne ceci:

    \begin{Verbatim}[commandchars=\\\{\}]
{\color{incolor}In [{\color{incolor}27}]:} string
\end{Verbatim}


    'bonjour'

    
    Afin de donner une valeur de type \emph{string} à une variable, on peut
utiliser les doubles `apostrophes', ou des ``guillemets''

    On peut assigner une valeur booléenne à une variable par \texttt{TRUE}
ou \texttt{FALSE}, mais aussi par simplement \texttt{T} ou \texttt{F}

    \begin{Verbatim}[commandchars=\\\{\}]
{\color{incolor}In [{\color{incolor}28}]:} booleen2 \PY{o}{\PYZlt{}\PYZhy{}}\PY{n+nb+bp}{T}
\end{Verbatim}


    \begin{Verbatim}[commandchars=\\\{\}]
{\color{incolor}In [{\color{incolor}29}]:} booleen2
\end{Verbatim}


    TRUE

    
    Il est possible de savoir quel type possède une variable gardée en
mémoire

    \begin{Verbatim}[commandchars=\\\{\}]
{\color{incolor}In [{\color{incolor}30}]:} \PY{k+kp}{class}\PY{p}{(}string\PY{p}{)}
\end{Verbatim}


    'character'

    
    \begin{Verbatim}[commandchars=\\\{\}]
{\color{incolor}In [{\color{incolor}31}]:} \PY{k+kp}{class}\PY{p}{(}booleen\PY{p}{)}
\end{Verbatim}


    'logical'

    
    \begin{Verbatim}[commandchars=\\\{\}]
{\color{incolor}In [{\color{incolor}32}]:} \PY{k+kp}{class}\PY{p}{(}num\PY{p}{)}
\end{Verbatim}


    'numeric'

    
    On peut aussi faire un test booléen sur le type d'une variable par\\
* \texttt{is.numeric(variable)} * \texttt{is.logical(variable)} *
\texttt{is.character(variable)}

    \begin{Verbatim}[commandchars=\\\{\}]
{\color{incolor}In [{\color{incolor}33}]:} \PY{k+kp}{is.logical}\PY{p}{(}num\PY{p}{)}
\end{Verbatim}


    FALSE

    
    \begin{Verbatim}[commandchars=\\\{\}]
{\color{incolor}In [{\color{incolor}34}]:} \PY{k+kp}{is.numeric}\PY{p}{(}num\PY{p}{)}
\end{Verbatim}


    TRUE

    
    \hypertarget{changement-du-type-de-variable}{%
\subsection{Changement du type de
variable}\label{changement-du-type-de-variable}}

    On peut aussi changer le type d'une variable

    \begin{Verbatim}[commandchars=\\\{\}]
{\color{incolor}In [{\color{incolor}35}]:} \PY{k+kp}{as.character}\PY{p}{(}num\PY{p}{)}
\end{Verbatim}


    '25'

    
    Remarquons les apostrophes. Toutefois, il faut faire attention avec les
conversions; essayons de convertir un \emph{string} en \emph{numeric}

    \begin{Verbatim}[commandchars=\\\{\}]
{\color{incolor}In [{\color{incolor}36}]:} \PY{c+c1}{\PYZsh{} as.numeric(string)}
\end{Verbatim}


    Nous obtenons \texttt{NA} (not available). Car R ne sait pas comment
traduire cette variable de type \emph{string} en \emph{numeric}.

    Toutefois, il est possible de changer des booléens vers numérique. *
TRUE=T=1 * FALSE=F=0

    \begin{Verbatim}[commandchars=\\\{\}]
{\color{incolor}In [{\color{incolor}37}]:} \PY{k+kp}{as.numeric}\PY{p}{(}booleen\PY{p}{)}
\end{Verbatim}


    1

    
    \hypertarget{tester-le-type-de-variable}{%
\subsection{Tester le type de
variable}\label{tester-le-type-de-variable}}

    On peut aussi faire un test booléen sur deux valeurs. Par exemple, on
peut demander si une valeur est plus petite ou égale (ou supérieure ou
égale) à une autre variable.

    \begin{Verbatim}[commandchars=\\\{\}]
{\color{incolor}In [{\color{incolor}38}]:} num\PY{o}{\PYZlt{}}\PY{l+m}{100}
\end{Verbatim}


    TRUE

    
    \begin{Verbatim}[commandchars=\\\{\}]
{\color{incolor}In [{\color{incolor}39}]:} num\PY{o}{\PYZlt{}=}\PY{l+m}{100}
\end{Verbatim}


    TRUE

    
    \begin{Verbatim}[commandchars=\\\{\}]
{\color{incolor}In [{\color{incolor}40}]:} num\PY{o}{\PYZgt{}=}\PY{l+m}{100}
\end{Verbatim}


    FALSE

    
    \hypertarget{test-duxe9galituxe9}{%
\subsubsection{Test d'égalité}\label{test-duxe9galituxe9}}

    Le test sur l'égalité se fait par un double ==

    \begin{Verbatim}[commandchars=\\\{\}]
{\color{incolor}In [{\color{incolor}41}]:} x\PY{o}{==}y
\end{Verbatim}


    TRUE

    
    \begin{Verbatim}[commandchars=\\\{\}]
{\color{incolor}In [{\color{incolor}42}]:} x\PY{o}{==}num
\end{Verbatim}


    FALSE

    
    \hypertarget{test-dinuxe9galituxe9}{%
\subsubsection{Test d'inégalité}\label{test-dinuxe9galituxe9}}

    Pour ce qui est du test d'inégalité, on utilise \texttt{!=}

    \begin{Verbatim}[commandchars=\\\{\}]
{\color{incolor}In [{\color{incolor}43}]:} x\PY{o}{!=}y
\end{Verbatim}


    FALSE

    
    \begin{Verbatim}[commandchars=\\\{\}]
{\color{incolor}In [{\color{incolor}44}]:} x\PY{o}{!=}num
\end{Verbatim}


    TRUE

    
    On peut aussi faire des tests logiques sur les valeurs de type string

    \begin{Verbatim}[commandchars=\\\{\}]
{\color{incolor}In [{\color{incolor}45}]:} string2\PY{o}{\PYZlt{}\PYZhy{}}\PY{l+s}{\PYZdq{}}\PY{l+s}{bonjours\PYZdq{}}
\end{Verbatim}


    \begin{Verbatim}[commandchars=\\\{\}]
{\color{incolor}In [{\color{incolor}46}]:} string\PY{o}{==}string2
\end{Verbatim}


    FALSE

    
    \begin{Verbatim}[commandchars=\\\{\}]
{\color{incolor}In [{\color{incolor}47}]:} string\PY{o}{!=}string2
\end{Verbatim}


    TRUE

    
    \hypertarget{les-opuxe9rations-sur-le-workspace}{%
\section{\texorpdfstring{Les opérations sur le
\texttt{workspace}}{Les opérations sur le workspace}}\label{les-opuxe9rations-sur-le-workspace}}

    \hypertarget{ruxe9pertoire-courant}{%
\subsection{Répertoire courant}\label{ruxe9pertoire-courant}}

    Afin de savoir dans quel répertoire nous travaillons, on peut utiliser
\texttt{getwd()}

    \begin{Verbatim}[commandchars=\\\{\}]
{\color{incolor}In [{\color{incolor}48}]:} \PY{k+kp}{getwd}\PY{p}{(}\PY{p}{)}
\end{Verbatim}


    '/Users/nour/MEGA/Studies/ACT3035/AUT\_2018'

    
    On peut également voir dans l'onglet \emph{files} à droite de l'écran
dans RStudio le répertoire dans lequel on travaille. Pour ceux qui sont
dans la version Linux, il suffit de taper \texttt{pwd} dans l'onglet
\emph{terminal}

    On remarque que la fonction \texttt{getwd()} nous retourne une valeur de
type string, on peut alors assigner cette valeur à une variable, par
exemple:

    \begin{Verbatim}[commandchars=\\\{\}]
{\color{incolor}In [{\color{incolor}49}]:} \PY{k+kp}{dir}\PY{o}{\PYZlt{}\PYZhy{}}\PY{k+kp}{getwd}\PY{p}{(}\PY{p}{)}
\end{Verbatim}


    \begin{Verbatim}[commandchars=\\\{\}]
{\color{incolor}In [{\color{incolor}50}]:} \PY{k+kp}{dir}
\end{Verbatim}


    '/Users/nour/MEGA/Studies/ACT3035/AUT\_2018'

    
    \hypertarget{changer-le-ruxe9pertoire-courant}{%
\subsection{Changer le répertoire
courant}\label{changer-le-ruxe9pertoire-courant}}

    On peut aussi changer le répertoire courant avec la fonction
\texttt{setwd("repertoire/sous-repertoire")}

    \begin{Verbatim}[commandchars=\\\{\}]
{\color{incolor}In [{\color{incolor}53}]:} \PY{k+kp}{setwd}\PY{p}{(}\PY{l+s}{\PYZsq{}}\PY{l+s}{/Users/nour/MEGA/Studies/ACT3035/AUT\PYZus{}2018\PYZsq{}}\PY{p}{)}
\end{Verbatim}


    \begin{Verbatim}[commandchars=\\\{\}]
{\color{incolor}In [{\color{incolor}54}]:} \PY{k+kp}{getwd}\PY{p}{(}\PY{p}{)}
\end{Verbatim}


    '/Users/nour/MEGA/Studies/ACT3035/AUT\_2018'

    
    On retourne à notre répertoire original, on se rappelle que la variable
\texttt{dir} contenant justement une valeur string du premier
répertoire. On peut réassigner une nouvelle valeur à notre répertoire
courant avec cette variable;

    \begin{Verbatim}[commandchars=\\\{\}]
{\color{incolor}In [{\color{incolor}55}]:} \PY{k+kp}{setwd}\PY{p}{(}\PY{k+kp}{dir}\PY{p}{)}
\end{Verbatim}


    \begin{Verbatim}[commandchars=\\\{\}]
{\color{incolor}In [{\color{incolor}56}]:} \PY{k+kp}{getwd}\PY{p}{(}\PY{p}{)}
\end{Verbatim}


    '/Users/nour/MEGA/Studies/ACT3035/AUT\_2018'

    
    Toutefois, il est aussi possible de le faire via l'IDE avec
\textbf{Session--\textgreater{} Set Working Directory --\textgreater{}
Choose Directory}.\\
OU avec le raccourci: \textbf{Ctrl+Shift+H}

    \hypertarget{lister-les-objets-dans-la-muxe9moire}{%
\subsection{Lister les objets dans la
mémoire}\label{lister-les-objets-dans-la-muxe9moire}}

    On peut avoir la liste des objets dans le répertoire courant avec la
fonction \texttt{ls()}. Comme dans les commandes Linux dans le terminal
(sans les parenthèses dans le cas de Linux)

    \begin{Verbatim}[commandchars=\\\{\}]
{\color{incolor}In [{\color{incolor}57}]:} \PY{k+kp}{ls}\PY{p}{(}\PY{p}{)}
\end{Verbatim}


    \begin{enumerate*}
\item 'booleen'
\item 'booleen2'
\item 'dir'
\item 'num'
\item 'p'
\item 'q'
\item 'string'
\item 'string2'
\item 'x'
\item 'y'
\item 'z'
\end{enumerate*}


    
    \hypertarget{supprimer-un-objets-de-la-muxe9moire}{%
\subsection{Supprimer un objets de la
mémoire}\label{supprimer-un-objets-de-la-muxe9moire}}

    Si l'on veut supprimer une variable, il suffit d'utiliser la fonction
\texttt{rm(variable)}

    \begin{Verbatim}[commandchars=\\\{\}]
{\color{incolor}In [{\color{incolor}58}]:} \PY{k+kp}{rm}\PY{p}{(}x\PY{p}{)}
\end{Verbatim}


    Regardons si x est encore là?

    \begin{Verbatim}[commandchars=\\\{\}]
{\color{incolor}In [{\color{incolor}59}]:} \PY{k+kp}{ls}\PY{p}{(}\PY{p}{)}
\end{Verbatim}


    \begin{enumerate*}
\item 'booleen'
\item 'booleen2'
\item 'dir'
\item 'num'
\item 'p'
\item 'q'
\item 'string'
\item 'string2'
\item 'y'
\item 'z'
\end{enumerate*}


    
    \hypertarget{vider-compluxe8tement-le-workspace}{%
\subsection{\texorpdfstring{Vider complètement le
\emph{workspace}}{Vider complètement le workspace}}\label{vider-compluxe8tement-le-workspace}}

    Maintenant, on peut aussi vider tout le \emph{workspace} avec;

    \begin{Verbatim}[commandchars=\\\{\}]
{\color{incolor}In [{\color{incolor}60}]:} \PY{k+kp}{rm}\PY{p}{(}\PY{k+kt}{list}\PY{o}{=}\PY{k+kp}{ls}\PY{p}{(}\PY{p}{)}\PY{p}{)}
\end{Verbatim}


    Il est aussi possible de le faire avec le ``ballet'' dans l'onglet
\emph{Environment}. Pour résumer, une fonction est simplement un appel à
un script créé auparavant. Certaines fonctions nécessitent des
arguments, et d'autres pas.

    \hypertarget{help}{%
\section{Help}\label{help}}

    \hypertarget{aide-sur-les-fonctions}{%
\subsection{Aide sur les fonctions}\label{aide-sur-les-fonctions}}

    Lorsqu'on ne se rappelle plus ce qu'une fonction fait, on peut appeler
cette fonction avec le caractère\texttt{?}précède le nom de la fonction.
Ça nous donne la documentation sur cette fonction.

    \begin{Verbatim}[commandchars=\\\{\}]
{\color{incolor}In [{\color{incolor}61}]:} \PY{o}{?}\PY{k+kp}{sqrt}
\end{Verbatim}


    L'autre façon d'avoir la documentation d'une fonction c'est de
simplement écrire \texttt{help(nomFonction)}

    \begin{Verbatim}[commandchars=\\\{\}]
{\color{incolor}In [{\color{incolor}62}]:} help\PY{p}{(}\PY{k+kp}{sqrt}\PY{p}{)}
\end{Verbatim}


    Si l'on ne se rappelle plus du nom exact de la fonction, on peut taper
ce qu'on pense être le nom de la fonction précédée de \texttt{??}

    \begin{Verbatim}[commandchars=\\\{\}]
{\color{incolor}In [{\color{incolor}63}]:} \PY{o}{?}\PY{o}{?}\PY{k+kp}{remove}
\end{Verbatim}


    
    
    Ça nous conduit vers l'onglet \emph{help} en faisant une recherche avec
le mot que nous avons tapé

    Le mot \texttt{base::} qui précède le nom de la fonction veut dire de
quel \emph{package} provient cette fonction. Dans notre exemple on peut
lire \texttt{base::rm}

    \hypertarget{la-fonction-example}{%
\subsection{\texorpdfstring{La fonction
\texttt{example}}{La fonction example}}\label{la-fonction-example}}

    Une autre fonction très utile \texttt{example(NomFonction)} qui nous
donne un exemple de la fonction que nous recherchons. En plus de nous
décrire les \emph{packages} nécessaires (qui sont chargés).

    \begin{Verbatim}[commandchars=\\\{\}]
{\color{incolor}In [{\color{incolor}64}]:} example\PY{p}{(}\PY{k+kp}{sqrt}\PY{p}{)}
\end{Verbatim}


    \begin{Verbatim}[commandchars=\\\{\}]

sqrt> require(stats) \# for spline

sqrt> require(graphics)

sqrt> xx <- -9:9

sqrt> plot(xx, sqrt(abs(xx)),  col = "red")

sqrt> lines(spline(xx, sqrt(abs(xx)), n=101), col = "pink")

    \end{Verbatim}

    \begin{center}
    \adjustimage{max size={0.9\linewidth}{0.9\paperheight}}{output_157_1.png}
    \end{center}
    { \hspace*{\fill} \\}
    
    \begin{Verbatim}[commandchars=\\\{\}]
{\color{incolor}In [{\color{incolor}65}]:} example\PY{p}{(}\PY{k+kp}{exp}\PY{p}{)}
\end{Verbatim}


    \begin{Verbatim}[commandchars=\\\{\}]

exp> log(exp(3))
[1] 3

exp> log10(1e7) \# = 7
[1] 7

exp> x <- 10\^{}-(1+2*1:9)

exp> cbind(x, log(1+x), log1p(x), exp(x)-1, expm1(x))
          x                                                    
 [1,] 1e-03 9.995003e-04 9.995003e-04 1.000500e-03 1.000500e-03
 [2,] 1e-05 9.999950e-06 9.999950e-06 1.000005e-05 1.000005e-05
 [3,] 1e-07 1.000000e-07 1.000000e-07 1.000000e-07 1.000000e-07
 [4,] 1e-09 1.000000e-09 1.000000e-09 1.000000e-09 1.000000e-09
 [5,] 1e-11 1.000000e-11 1.000000e-11 1.000000e-11 1.000000e-11
 [6,] 1e-13 9.992007e-14 1.000000e-13 9.992007e-14 1.000000e-13
 [7,] 1e-15 1.110223e-15 1.000000e-15 1.110223e-15 1.000000e-15
 [8,] 1e-17 0.000000e+00 1.000000e-17 0.000000e+00 1.000000e-17
 [9,] 1e-19 0.000000e+00 1.000000e-19 0.000000e+00 1.000000e-19

    \end{Verbatim}

    \hypertarget{aide-sur-les-donnuxe9es}{%
\subsection{Aide sur les données}\label{aide-sur-les-donnuxe9es}}

    Il est aussi possible d'avoir plus d'informations sur les données
préchargées.

    \begin{Verbatim}[commandchars=\\\{\}]
{\color{incolor}In [{\color{incolor}66}]:} data\PY{p}{(}\PY{p}{)}
\end{Verbatim}


    Ce sont des bases de données de R qui sont disponibles par défaut

    \begin{Verbatim}[commandchars=\\\{\}]
{\color{incolor}In [{\color{incolor}67}]:} plot\PY{p}{(}cars\PY{p}{)}
\end{Verbatim}


    \begin{center}
    \adjustimage{max size={0.9\linewidth}{0.9\paperheight}}{output_163_0.png}
    \end{center}
    { \hspace*{\fill} \\}
    
    On peut aussi charger des données déjà disponibles dans un
\emph{package}

    \begin{Verbatim}[commandchars=\\\{\}]
{\color{incolor}In [{\color{incolor}68}]:} \PY{k+kn}{require}\PY{p}{(}MASS\PY{p}{)}
\end{Verbatim}


    \begin{Verbatim}[commandchars=\\\{\}]
Loading required package: MASS

    \end{Verbatim}

    \begin{Verbatim}[commandchars=\\\{\}]
{\color{incolor}In [{\color{incolor}69}]:} data\PY{p}{(}Cars93\PY{p}{)}
\end{Verbatim}


    Une fonction très utile afin d'avoir un sommaire rapide sur les données
est \texttt{summary}

    \begin{Verbatim}[commandchars=\\\{\}]
{\color{incolor}In [{\color{incolor}70}]:} \PY{k+kp}{summary}\PY{p}{(}Cars93\PY{p}{)}
\end{Verbatim}


    
    \begin{verbatim}
    Manufacturer     Model         Type      Min.Price         Price      
 Chevrolet: 8    100    : 1   Compact:16   Min.   : 6.70   Min.   : 7.40  
 Ford     : 8    190E   : 1   Large  :11   1st Qu.:10.80   1st Qu.:12.20  
 Dodge    : 6    240    : 1   Midsize:22   Median :14.70   Median :17.70  
 Mazda    : 5    300E   : 1   Small  :21   Mean   :17.13   Mean   :19.51  
 Pontiac  : 5    323    : 1   Sporty :14   3rd Qu.:20.30   3rd Qu.:23.30  
 Buick    : 4    535i   : 1   Van    : 9   Max.   :45.40   Max.   :61.90  
 (Other)  :57    (Other):87                                               
   Max.Price       MPG.city      MPG.highway                  AirBags  
 Min.   : 7.9   Min.   :15.00   Min.   :20.00   Driver & Passenger:16  
 1st Qu.:14.7   1st Qu.:18.00   1st Qu.:26.00   Driver only       :43  
 Median :19.6   Median :21.00   Median :28.00   None              :34  
 Mean   :21.9   Mean   :22.37   Mean   :29.09                          
 3rd Qu.:25.3   3rd Qu.:25.00   3rd Qu.:31.00                          
 Max.   :80.0   Max.   :46.00   Max.   :50.00                          
                                                                       
 DriveTrain  Cylinders    EngineSize      Horsepower         RPM      
 4WD  :10   3     : 3   Min.   :1.000   Min.   : 55.0   Min.   :3800  
 Front:67   4     :49   1st Qu.:1.800   1st Qu.:103.0   1st Qu.:4800  
 Rear :16   5     : 2   Median :2.400   Median :140.0   Median :5200  
            6     :31   Mean   :2.668   Mean   :143.8   Mean   :5281  
            8     : 7   3rd Qu.:3.300   3rd Qu.:170.0   3rd Qu.:5750  
            rotary: 1   Max.   :5.700   Max.   :300.0   Max.   :6500  
                                                                      
  Rev.per.mile  Man.trans.avail Fuel.tank.capacity   Passengers   
 Min.   :1320   No :32          Min.   : 9.20      Min.   :2.000  
 1st Qu.:1985   Yes:61          1st Qu.:14.50      1st Qu.:4.000  
 Median :2340                   Median :16.40      Median :5.000  
 Mean   :2332                   Mean   :16.66      Mean   :5.086  
 3rd Qu.:2565                   3rd Qu.:18.80      3rd Qu.:6.000  
 Max.   :3755                   Max.   :27.00      Max.   :8.000  
                                                                  
     Length        Wheelbase         Width        Turn.circle   
 Min.   :141.0   Min.   : 90.0   Min.   :60.00   Min.   :32.00  
 1st Qu.:174.0   1st Qu.: 98.0   1st Qu.:67.00   1st Qu.:37.00  
 Median :183.0   Median :103.0   Median :69.00   Median :39.00  
 Mean   :183.2   Mean   :103.9   Mean   :69.38   Mean   :38.96  
 3rd Qu.:192.0   3rd Qu.:110.0   3rd Qu.:72.00   3rd Qu.:41.00  
 Max.   :219.0   Max.   :119.0   Max.   :78.00   Max.   :45.00  
                                                                
 Rear.seat.room   Luggage.room       Weight         Origin              Make   
 Min.   :19.00   Min.   : 6.00   Min.   :1695   USA    :48   Acura Integra: 1  
 1st Qu.:26.00   1st Qu.:12.00   1st Qu.:2620   non-USA:45   Acura Legend : 1  
 Median :27.50   Median :14.00   Median :3040                Audi 100     : 1  
 Mean   :27.83   Mean   :13.89   Mean   :3073                Audi 90      : 1  
 3rd Qu.:30.00   3rd Qu.:15.00   3rd Qu.:3525                BMW 535i     : 1  
 Max.   :36.00   Max.   :22.00   Max.   :4105                Buick Century: 1  
 NA's   :2       NA's   :11                                  (Other)      :87  
    \end{verbatim}

    
    Ça nous donne les variables trouvées dans cette base de données ainsi
qu'une statistique descriptive sur chacune des variables

\begin{longtable}[]{@{}l@{}}
\toprule
Min.\tabularnewline
\midrule
\endhead
1st Qu.\tabularnewline
Median\tabularnewline
Mean\tabularnewline
3rd Qu.\tabularnewline
Max.\tabularnewline
\bottomrule
\end{longtable}

    \begin{Verbatim}[commandchars=\\\{\}]
{\color{incolor}In [{\color{incolor}71}]:} \PY{k+kp}{head}\PY{p}{(}Cars93\PY{p}{)}
\end{Verbatim}


    \begin{tabular}{r|lllllllllllllllllllllllllll}
 Manufacturer & Model & Type & Min.Price & Price & Max.Price & MPG.city & MPG.highway & AirBags & DriveTrain & ⋯ & Passengers & Length & Wheelbase & Width & Turn.circle & Rear.seat.room & Luggage.room & Weight & Origin & Make\\
\hline
	 Acura              & Integra            & Small              & 12.9               & 15.9               & 18.8               & 25                 & 31                 & None               & Front              & ⋯                  & 5                  & 177                & 102                & 68                 & 37                 & 26.5               & 11                 & 2705               & non-USA            & Acura Integra     \\
	 Acura                & Legend               & Midsize              & 29.2                 & 33.9                 & 38.7                 & 18                   & 25                   & Driver \& Passenger & Front                & ⋯                    & 5                    & 195                  & 115                  & 71                   & 38                   & 30.0                 & 15                   & 3560                 & non-USA              & Acura Legend        \\
	 Audi               & 90                 & Compact            & 25.9               & 29.1               & 32.3               & 20                 & 26                 & Driver only        & Front              & ⋯                  & 5                  & 180                & 102                & 67                 & 37                 & 28.0               & 14                 & 3375               & non-USA            & Audi 90           \\
	 Audi                 & 100                  & Midsize              & 30.8                 & 37.7                 & 44.6                 & 19                   & 26                   & Driver \& Passenger & Front                & ⋯                    & 6                    & 193                  & 106                  & 70                   & 37                   & 31.0                 & 17                   & 3405                 & non-USA              & Audi 100            \\
	 BMW                & 535i               & Midsize            & 23.7               & 30.0               & 36.2               & 22                 & 30                 & Driver only        & Rear               & ⋯                  & 4                  & 186                & 109                & 69                 & 39                 & 27.0               & 13                 & 3640               & non-USA            & BMW 535i          \\
	 Buick              & Century            & Midsize            & 14.2               & 15.7               & 17.3               & 22                 & 31                 & Driver only        & Front              & ⋯                  & 6                  & 189                & 105                & 69                 & 41                 & 28.0               & 16                 & 2880               & USA                & Buick Century     \\
\end{tabular}


    
    \hypertarget{plus-de-resources}{%
\section{Plus de resources}\label{plus-de-resources}}

    \begin{itemize}
\item
  \textbf{The R Project for Statistical Computing}:
  (http://www.r-project.org/) Premier lieu où.
\item
  \textbf{The Comprehensive R Archive Network}:
  (http://cran.r-project.org/) C,est là ou se trouve le logiciel R, avec
  des miliers de \emph{packages}, il s,y trouve aussi des exemples et
  même des livres!
\item
  \textbf{R-Forge}: (http://r-forge.r-project.org/) Une autre place où
  des \emph{packages} sont sauvegardé, on y trouve aussi des
  \emph{packages} tout récemment développés
\item
  \textbf{Rlanguage reddit}: (https://www.reddit.com/r/Rlanguage) On y
  trouve toutes sortes d'informations ou question
  \href{https://www.reddit.com/r/Rlanguage/comments/786y4z/recommended_books/}{exemple}
\end{itemize}


    % Add a bibliography block to the postdoc
    
    
    
    \end{document}
